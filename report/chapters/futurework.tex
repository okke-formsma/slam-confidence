%!TEX root = ../report.tex

\chapter{Discussion}
\label{discussion}

** This is only a list of points that need to be discussed, need to be fleshed out **

It is easy to break the map where the scanmatcher fails - it fails most significantly when it can not find a single matching scan. See also part about ins sensor.

Scanmatcher works too well already - the hough stitching is an inferior solution.

Locating multiple robots might be a better use of the scanmatcher than stiching the map of one robot.

The rotation estimate works very well in an indoor environment, this could be used as input for the scanmatcher.

----

Ins sensor: should take it's rotation drift into consideration when assessing the next XY coordinates. Otherwise, update the location and rotation by the change in the sensor instead of the full values.

\section{Future work}
\label{futurework}

** This is only a list of points that need to be discussed, need to be fleshed out **

Find a better X- Y- matching algorithm for the Hough stitching method

Use the quad-tree based scanmatcher, and maybe use this mechanism to stitch maps from multiple robots. Or at least to align them.

Fix the INS sensor, because it is not internally coherent. (Or: don't trust the rotation by the INS sensor)

