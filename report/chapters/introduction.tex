%!TEX root = ../report.tex
\chapter{Introduction}
\label{chapter:introduction}
A compelling use of robotics is in the Urban Search And Rescue setting. The goal of Urban Search And Rescue robot teams are to help emergency services find victims of disasters, and perform duties which are deemed to dangerous for humans\footnote{\url{http://usarsim.sourceforge.net/wiki/index.php/UT_3_Manual}}. In many disaster areas, human responders may be at risk due to fire, smoke, toxicity, radioactivity or the possiblility of a secondary disaster. A team of robots is much better suited to explore such dangerous disaster areas than humans. 

Urban Search And Rescue robots face many challenges. They must navigate unknown environments with many potential hazards. The problem of mapping an unknown environment and localization within this environment is known as SLAM, for Simultaneous Localization And Mapping \cite{durrant2006simultaneous, bailey2006simultaneous}. Testing such algorithms with actual robots in real-life environments is expensive and cumbersome. Robots and sensors are expensive, as well as creating a disaster-environment. 

A high fidelity simulation environment based on the Unreal Engine called USARSim\footnote{Urban Search and Rescue simulation website: \url{http://sourceforge.net/apps/mediawiki/usarsim/index.php?title=Main_Page}} \cite{balaguer2008usarsim} (Urban Search And Rescue Simulation) provides an platform on which to test the performance of the robots in a disaster environment without the cost and effort associated with real robots. These environments can both be indoors and outdoors, and are typically not larger than a housing block. A team of robots is deployed in this environment with the task of localizing (human) victims. One of the goals of the USARSim project is to improve autonomous behavior, allowing one human operator to control a team of up to 16 robots simultaneously. 

The RoboCup international robotics competition utilizes USARSim in their Rescue Simulation League. This competition draws teams from universities all over the world to show their advances in robotic control software, focusing on issues such as human-robot interfaces, autonomous behavior and team coordination. Much work has been done to increase the realism of the simulation. A selection of recent work: an assessment for use of USARSim in Human Robot Interaction \cite{wang2005validating}, for use with walking robots \cite{van2012validation}, adding realistic laser sensor behavior in environments with smoke \cite{formsma2011realistic} and creating a omnidirectional camera sensor \cite{schmits2009omnidirectional}.

Teams from all over the world compete in the annual RoboCup Rescue Simulation competition\footnote{RoboCup general website \url{http://www.robocup.org/}}. The universities of Oxford and Amsterdam participated until 2011 with a cooperative team: the Amsterdam Oxford Joint Rescue Forces \cite{aojrf2011, visser2012uva}. The robot control software used by this team is UsarCommander\footnote{Amsterdam Oxford Joint Rescue Forces website \url{http://www.jointrescueforces.eu/}}. UsarCommander contains a number of SLAM implementations, including ManifoldSLAM. The default Simulatenous Localization And Mapping (SLAM) method employed by the AOJRF is not robust against sensor failures, which occur for example when the robot (with the laser scanner) is severly tilted. 

In this report a post-processing map stitching method is proposed which performs a global optimization over the map created by an online SLAM method. A novel method which splits a map when the localization uncertainty exceeds a certain threshold is presented. The different pieces of the map are then stitched together using a Hough transform based map stitching method. The dataset for the experiments consists of sensor data acquired by a robot in the USARSim environment. 

This report is structured as follows. In chapter~\ref{slam} an overview of the Simultaneous Localization and Mapping problem and related algorithms is provided. Chapter~\ref{chapter:hough} presents the Hough-based stitching method. The experimental method is described in chapter~\ref{method}, and experimental results in chapter~\ref{experiments}. The results are discussed in chapter~\ref{discussion} and pointers for future work in chapter~\ref{futurework}.